%%
% Plantilla de Trabajo
% Modificación de una plantilla de Latex de Frits Wenneker para adaptarla 
% al castellano y a las necesidades de escribir informática y matemáticas.
%
% Editada por: Mario Román
%
% License:
% CC BY-NC-SA 3.0 (http://creativecommons.org/licenses/by-nc-sa/3.0/)
%%

%%%%%%%%%%%%%%%%%%%%
% Short Sectioned Assignment
% LaTeX Template
% Version 1.0 (5/5/12)
%
% This template has been downloaded from:
% http://www.LaTeXTemplates.com
%
% Original author:
% Frits Wenneker (http://www.howtotex.com)
%
% License:
% CC BY-NC-SA 3.0 (http://creativecommons.org/licenses/by-nc-sa/3.0/)
%
%%%%%%%%%%%%%%%%%%%%%

%----------------------------------------------------------------------------------------
%	PAQUETES Y CONFIGURACIÓN DEL DOCUMENTO
%----------------------------------------------------------------------------------------

%% Configuración del papel.
% fourier: Usa la fuente Adobe Utopia. (Comentando la línea usa la fuente normal)
\documentclass[paper=a4, fontsize=11pt, spanish]{scrartcl} 
\usepackage{fourier}

% Centra y formatea los títulos de sección.
% Quita la indentación de párrafos.
\usepackage{sectsty} % Allows customizing section commands
\allsectionsfont{\centering \normalfont\scshape} % Make all sections centered, the default font and small caps
\setlength\parindent{0pt} % Removes all indentation from paragraphs - comment this line for an assignment with lots of text

%% Castellano.
% noquoting: Permite uso de comillas no españolas.
% lcroman: Permite la enumeración con numerales romanos en minúscula.
% fontenc: Usa la fuente completa para que pueda copiarse correctamente del pdf.
\usepackage[spanish,es-noquoting,es-lcroman]{babel}
\usepackage[utf8]{inputenc}
\usepackage[T1]{fontenc}
\selectlanguage{spanish}

%% Matemáticas.
% Paquetes de la AMS. Para entornos de ecuaciones.
\usepackage{amsmath,amsfonts,amsthm}

% Enlaces
\usepackage[hidelinks]{hyperref}

%----------------------------------------------------------------------------------------
%	TÍTULO
%----------------------------------------------------------------------------------------
% Título con las líneas horizontales, nombres y fecha.

\newcommand{\horrule}[1]{\rule{\linewidth}{#1}} % Create horizontal rule command with 1 argument of height

\title{
  \normalfont \normalsize 
  \textsc{Universidad de Granada.} \\ [25pt] % Your university, school and/or department name(s)
  \horrule{0.5pt} \\[0.4cm] % Thin top horizontal rule
  \huge Ejercicios de Álgebra, Grupos y Representaciones \\ % The assignment title
  \horrule{2pt} \\[0.5cm] % Thick bottom horizontal rule
}

\author{Óscar Bermúdez Garrido} % Your name

\date{\normalsize\today} % Today's date or a custom date

%----------------------------------------------------------------------------------------
%	DOCUMENTO
%----------------------------------------------------------------------------------------


\begin{document}
	\maketitle % Escribe el título
	
	\newpage

	\begin{enumerate}
		% Pág 17
		\item Sea $\mathbb{P}_n$ el espacio vectorial real de las funciones polinómicas en una variable de grado
		menor o igual que $n$. Sea $T: \mathbb{P}_n \rightarrow \mathbb{P}_n$ la aplicación lineal que asigna a
		cada polinomio su derivada. Calcular una serie de composición de $\mathbb{P}_n$ visto como $\mathbb{R}[X]
		-módulo$ vía $T$.
		\subsubsection*{Solución}
		Sabemos que una base del espacio de polinomio de grado menor o igual que $k$ sería $M_k = \{1, \dots, x^k\}$.
		Claramente, podemos ver que $<M_k> = <\{1, \dots, x^k\}>$ es un submódulo de $\mathbb{P}_n$ cerrado para
		la derivada puesto que $M_k$ es base del espacio de polinomios de grado menor o igual que $k$ y, al estar
		en un cuerpo de característica 0, su derivada es de grado menor o igual que $k-1$ y claramente $M_{k-1}
		\subseteq <M_k>$.
		
		Además, dado que $M_k = <x^k> M_k/_{M_{k-1}}$ se tiene que $M_k$ es de dimensión 1 en el cuerpo de los
		reales sobre $M_k/_{M_{k-1}}$ y por tanto simple.
		
		Así formaríamos la serie de descomposición: $0 \subset <M_1> \subset <M_2> \subset \dots \subset <M_n> =
		\mathbb{P}_n$.
		
		% Pág 17
		\item ** En las condiciones del ejercicio anterior, calcular todos los $\mathbb{R}-submódulos$ de
		$\mathbb{P}_n$.
		\subsubsection*{Solución}
		En el ejercicio anterior vimos los módulos de la forma $<M_k>$, que eran los que formaban la serie de
		descomposición que dimos. Veamos ahora que estos son los únicos submódulos de $\mathbb{P}_n$.
		
		Supongamos que existen más submódulos de $\mathbb{P}_n$, entonces tomemos un polinomio $p(x) = x^k +
		a_{k-1}x^{k-1} + \cdots + a_1x+a_0$ con $k \leq n \Rightarrow p(x) \in \mathbb{P}_n$ y sea $N_p$ el menor
		submódulo tal que $p(x) \in N_p$.
		
		Como $N_p$ es un submódulo debe ser cerrado para $T$, luego se tiene
		que $\delta p(x) = kx^{k-1} + (k-1)a_{k-1}x^{k-2} + \cdots + 2a_2x+a_1 \in N_p$ y análogamente se deduce
		que$\delta^2 p(x), \dots, \delta^{k} \in N_p$, es decir, $<\{p, \delta p, \delta^2 p, \dots, \delta^k p\}>
		\subseteq N_p$.
		
		Entonces, es inmediato comprobar que $ <M_k> \cong <\{p, \delta p, \delta^2 p, \dots, \delta^k p\}>$, y
		se tiene que $<M_k> \subseteq N_p$ pero por cómo hemos definido $N_p$ se deduce que $<M_k> = N_p$.
		
		% Pág 22
		\item ** Sea $R$ un álgebra sobre un cuerpo de característica distinta de 2, y $a, b, e \in R$ idempotentes.
		Demostrar que si $e = a + b$, entonces $ab = ba = 0$. Si la característica es 2, encontrar un contraejemplo
		con $a \neq b$.
		\subsubsection*{Solución}
		Si $e = a+b$ tenemos que $e^2 = (a+b)^2 = a^2 + b^2 + ab + ba = a + b + ab + ba = a+b \Rightarrow ab + ba
		= 0 \Rightarrow ba = -ab \Leftrightarrow ab = -ba$ y volviendo a utilizar que $a$ y $b$ son idempotentes,
		se llega a que $ab = a^2b = a(-ba) = -aba = (-ab)a = ba^2 = ba \Rightarrow 2ab = 2ba = ab + ba = 0$ pero
		como la característica del cuerpo no es 2, la única posibilidad es que $ab = ba = 0$.
		
		Como contraejemplo, nos iremos al álgebra $M_2(\mathbb{F}_2)$ y tomaremos $a = I_2$, $\displaystyle b =
		\begin{pmatrix} 1 & 0 \\ 0 & 0 \end{pmatrix}$, entonces $\displaystyle e = a+b = \begin{pmatrix} 0 & 0
		\\ 0 & 1 \end{pmatrix}$, claramente todas son idempotentes pero $ab = ba = b \neq 0_{2,2}$.
	
		% Pág 43
		\item Si $r \in \mathbb{Q}$ es un entero algebraico, entonces $r \in \mathbb{Z}$\footnote{En el enunciado
		original, era una $q$ pero por conveniencia de notación, la he cambiado.}.
		\subsubsection*{Solución}
		Supongamos que $r$ es un entero algebraico pero $r \notin \mathbb{Z}$ y sea $f(x) = x^n + a_{n-1}x^{n-1}
		+ \dots + a_1x + a_0$ con $a_{n-1}, \dots, a_1, a_0 \in \mathbb{Z}$ el polinomio de coeficientes enteros
		del que $r$ es raíz, esto es, $f(r) = 0$.
		
		Como $r \in \mathbb{Q}$ podemos expresarlo como $\displaystyle r = \frac{p}{q}$ con $(p, q) = 1$, entonces
		nos queda que $\displaystyle f(r) = f\left(\frac{p}{q}\right) = \left(\frac{p}{q}\right)^n + a_{n-1}
		\left(\frac{p}{q}\right)^{n-1} + \dots + a_1\frac{p}{q} + a_0 = 0$.
		
		Después, multiplicamos toda la expresión por $q^n$ para quitarnos los denominadores: $p^n + a_{n-1}qp^{n-1}
		+ \dots + a_1q^{n-1}p + a_0q^n = 0$
		
		Llegados a este punto, vemos que $-p^n = q \cdot \left(a_{n-1}p^{n-1} + \dots + a_1q^{n-2}p + a_0q^{n-1}\right)
		\Rightarrow q$ divide a $p^n$ pero recordemos que $(p, q) = 1$.
	\end{enumerate}
\end{document}
