%%
% Plantilla de Trabajo
% Modificación de una plantilla de Latex de Frits Wenneker para adaptarla 
% al castellano y a las necesidades de escribir informática y matemáticas.
%
% Editada por: Mario Román
%
% License:
% CC BY-NC-SA 3.0 (http://creativecommons.org/licenses/by-nc-sa/3.0/)
%%

%%%%%%%%%%%%%%%%%%%%
% Short Sectioned Assignment
% LaTeX Template
% Version 1.0 (5/5/12)
%
% This template has been downloaded from:
% http://www.LaTeXTemplates.com
%
% Original author:
% Frits Wenneker (http://www.howtotex.com)
%
% License:
% CC BY-NC-SA 3.0 (http://creativecommons.org/licenses/by-nc-sa/3.0/)
%
%%%%%%%%%%%%%%%%%%%%%

%----------------------------------------------------------------------------------------
%	PAQUETES Y CONFIGURACIÓN DEL DOCUMENTO
%----------------------------------------------------------------------------------------

%% Configuración del papel.
% fourier: Usa la fuente Adobe Utopia. (Comentando la línea usa la fuente normal)
\documentclass[paper=a4, fontsize=11pt, spanish]{scrartcl} 
\usepackage{fourier}

% Centra y formatea los títulos de sección.
% Quita la indentación de párrafos.
\usepackage{sectsty} % Allows customizing section commands
\allsectionsfont{\centering \normalfont\scshape} % Make all sections centered, the default font and small caps
\setlength\parindent{0pt} % Removes all indentation from paragraphs - comment this line for an assignment with lots of text

%% Castellano.
% noquoting: Permite uso de comillas no españolas.
% lcroman: Permite la enumeración con numerales romanos en minúscula.
% fontenc: Usa la fuente completa para que pueda copiarse correctamente del pdf.
\usepackage[spanish,es-noquoting,es-lcroman]{babel}
\usepackage[utf8]{inputenc}
\usepackage[T1]{fontenc}
\selectlanguage{spanish}

% Para incluir imágenes y colocarlas
\usepackage{graphics,graphicx, float, url}

%% Matemáticas.
% Paquetes de la AMS. Para entornos de ecuaciones.
\usepackage{amsmath,amsfonts,amsthm}

% Enlaces
\usepackage[hidelinks]{hyperref}

%----------------------------------------------------------------------------------------
%	TÍTULO
%----------------------------------------------------------------------------------------
% Título con las líneas horizontales, nombres y fecha.

\newcommand{\horrule}[1]{\rule{\linewidth}{#1}} % Create horizontal rule command with 1 argument of height

\title{
  \normalfont \normalsize 
  \textsc{Universidad de Granada.} \\ [25pt] % Your university, school and/or department name(s)
  \horrule{0.5pt} \\[0.4cm] % Thin top horizontal rule
  \huge Ejercicios de Álgebra, Grupos y Representaciones \\ % The assignment title
  \horrule{2pt} \\[0.5cm] % Thick bottom horizontal rule
}

\author{Óscar Bermúdez Garrido} % Your name

\date{\normalsize\today} % Today's date or a custom date

%----------------------------------------------------------------------------------------
%	DOCUMENTO
%----------------------------------------------------------------------------------------


\begin{document}
	\maketitle % Escribe el título
	
	\newpage

	\begin{enumerate}
		% Pág 6
		\item * Calcular el centro del álgebra de matrices $M_n(K)$.
		\begin{proof}[Solución]
			Sea $C \in Z(M_n(K))$ y sean las matrices $E_{kl} = (\delta_{kl})_{i,j}$, es decir, las matrices con
			1 en la posición $(k, l)$ y 0 en el resto.
			
			Entonces, se tiene $CE_{ij} =
			\begin{pmatrix}
				   0   &    0   & \dots  & {}^{\left.j\right)}c_{1i} & \dots  &    0   \\
				   0   &    0   & \dots  &           c_{2i}          & \dots  &    0   \\
				\vdots & \vdots &        &           \vdots          &        & \vdots \\
				   0   &    0   & \dots  &           c_{ni}          & \dots  &    0
			\end{pmatrix}$
			y $E_{ij}C =
			\begin{pmatrix}
				             0            &    0   & \dots  &    0   \\
				             0            &    0   & \dots  &    0   \\
				          \vdots          & \vdots &        & \vdots \\
				{}^{\left.i\right)}c_{j1} & c_{j2} & \dots  & c_{jn} \\
				          \vdots          & \vdots &        & \vdots \\
				             0            &    0   & \dots  &    0
			\end{pmatrix}$
			
			Y claramente, se deduce que $CE_{ij} = E_{ij}C \Leftrightarrow
			\begin{cases}
				c_{jj} = c_{ii} \\
				c_{ij} = 0 \qquad \forall i \neq j
			\end{cases}$
			
			Dado que esta relación se debe cumplir $\forall i,j \in \{1, 2, \dots, n\}$, podemos concluir que
			$C = c_{11}I_n = c_{22}I_n = \dots = c_{nn}I_n = cI_n$. Notemos además que $c$ puede ser cualquier
			elemento de $K$ ya que $\forall A \in M_n(K)$ se tiene que $cI_nA = cA = Ac = AcI_n$.
			
			Por lo que, finalmente, hemos demostrado que $Z(M_n(K)) = \{kI_n: k \in K\}$.
		\end{proof}
		
		% Pág 15
		\item Sea $M$ un $A$-módulo. Demostrar que $M$ es simple $\Leftrightarrow M = Am \quad \forall 0 \neq m
		\in M$.
		\begin{proof}
			\fbox{$\Rightarrow$} Tomamos $0 \neq m \in M$ Claramente, $Am$ es un submódulo no nulo de $M$, esto
			es, $0 \subset Am \subseteq M$ pero como $M$ es simple, se tiene que $Am = M$. Luego, $M$ simple
			$\Rightarrow Am = M \quad \forall 0 \neq m \in M$.
			
			\fbox{$\Leftarrow$} Sea $0 \subset N \subseteq M$ submódulo, tomamos entonces $0 \neq n \in N$.
			Como $N$ es un submódulo, tiene que ser cerrado para producto por elementos de $A \Rightarrow An
			\subseteq N$. Además, como $n \in N \subseteq M$, se tiene que $n$ cumple que $An = M$. Finalmente,
			concluimos que $M = An \subseteq N \subseteq M \Rightarrow N = M \Rightarrow M$ es simple.
		\end{proof}
		
		% Pág 17
		\item * Consideramos $T: \mathbb{R}^3 \rightarrow \mathbb{R}^3$ una aplicación lineal, y la estructura
		de $\mathbb{R}[X]$-módulo correspondiente sobre $\mathbb{R}^3$. Discutir los posibles valores de la
		longitud de $\mathbb{R}^3$ como $\mathbb{R}$-módulo. Poner un ejemplo de $T$ para el que se alcance
		cada longitud.
		\begin{proof}[Solución]
			Dado que la dimensión de $\mathbb{R}^3$ como espacio vectorial es 3, su longitud será, como máximo, 3.
			
			\begin{itemize}
				\item \textbf{Longitud 1:} $0 \subseteq \mathbb{R}^3$, es decir, sería simple.
				
				Sin embargo, este caso no puede darse pues toda aplicación lineal $T: \mathbb{R}^3 \rightarrow
				\mathbb{R}^3$ nos da una ecuación polinómica de grado 3 con coeficientes reales $|T - \lambda I|
				= 0$ que tiene una solución en los reales. Por tanto, obtenemos así un subespacio estricto que
				queda fijo por $T$, es decir, un submódulo $\Rightarrow$ no es simple.
				
				\item \textbf{Longitud 2:} $0 \subseteq N \subseteq \mathbb{R}^3$
				
				En este caso, necesitamos encontrar una aplicación que mueva un elemento bidimensional mientras
				deja fija otro unideminensional. Por ejemplo, el giro sobre el eje (1, 0, 0):
				
				$$T = \begin{pmatrix}
					1 & 0 & 0 \\
					0 & 0 & 1 \\
					0 & -1 & 0
				\end{pmatrix}$$
				
				Claramente, el subespacio generado por el eje queda fijo por esta acción, así que se tiene que
				$N = <(0, 1, 0), (0, 0, 1)>$.
				
				\item \textbf{Longitud 3:} $0 \subseteq N_1 \subseteq N_2 \subseteq \mathbb{R}^3$
				
				Este caso es el más fácil de todos pues nos baste con tomar $T = I_3$ para tener la descomposicón
				$0 \subseteq <(1, 0, 0)> \subseteq <(1, 0, 0), (0, 1, 0)> \subseteq \mathbb{R}^3$.
			\end{itemize}
		\end{proof}
		
		% Pág 17
		\item Sea $\mathbb{P}_n$ el espacio vectorial real de las funciones polinómicas en una variable de grado
		menor o igual que $n$. Sea $T: \mathbb{P}_n \rightarrow \mathbb{P}_n$ la aplicación lineal que asigna a
		cada polinomio su derivada. Calcular una serie de composición de $\mathbb{P}_n$ visto como $\mathbb{R}[X]
		-módulo$ vía $T$.
		\begin{proof}[Solución]
			Sabemos que una base del espacio de polinomio de grado menor o igual que $k$ sería $M_k = \{1, \dots,
			x^k\}$. Claramente, podemos ver que $<M_k> = <\{1, \dots, x^k\}>$ es un submódulo de $\mathbb{P}_n$
			cerrado para la derivada puesto que $M_k$ es base del espacio de polinomios de grado menor o igual
			que $k$ y, al estar en un cuerpo de característica 0, su derivada es de grado menor o igual que $k-1$
			y claramente $M_{k-1} \subseteq <M_k>$.
			
			Además, dado que $M_k = <x^k> M_k/_{M_{k-1}}$ se tiene que $M_k$ es de dimensión 1 en el cuerpo de
			los reales sobre $M_k/_{M_{k-1}}$ y por tanto simple.
			
			Así formaríamos la serie de descomposición: $0 \subset <M_1> \subset <M_2> \subset \dots \subset
			<M_n> = \mathbb{P}_n$.
		\end{proof}
		
		% Pág 17
		\item ** En las condiciones del ejercicio anterior, calcular todos los $\mathbb{R}$-submódulos de
		$\mathbb{P}_n$.
		\begin{proof}[Solución]
			En el ejercicio anterior vimos los módulos de la forma $<M_k>$, que eran los que formaban la serie
			de descomposición que dimos. Veamos ahora que estos son los únicos submódulos de $\mathbb{P}_n$.
			
			Supongamos que existen más submódulos de $\mathbb{P}_n$, entonces tomemos un polinomio $p(x) = x^k
			+ a_{k-1}x^{k-1} + \cdots + a_1x+a_0$ con $k \leq n \Rightarrow p(x) \in \mathbb{P}_n$ y sea $N_p$
			el menor submódulo tal que $p(x) \in N_p$.
			
			Como $N_p$ es un submódulo debe ser cerrado para $T$, luego se tiene
			que $\delta p(x) = kx^{k-1} + (k-1)a_{k-1}x^{k-2} + \cdots + 2a_2x+a_1 \in N_p$ y análogamente se
			deduce que$\delta^2 p(x), \dots, \delta^{k} \in N_p$, es decir, $<\{p, \delta p, \delta^2 p, \dots,
			\delta^k p\}> \subseteq N_p$.
			
			Entonces, es inmediato comprobar que $ <M_k> \cong <\{p, \delta p, \delta^2 p, \dots, \delta^k p\}>$,
			y se tiene que $<M_k> \subseteq N_p$ pero por cómo hemos definido $N_p$ se deduce que $<M_k> = N_p$.
		\end{proof}
		
		% Pág 19
		\item ** Supongamos $T: V \rightarrow V$ un endomorfismo $K$-lineal, donde $V$ es un espacio vectorial de
		dimensión finita que consideramos, como de costumbre, como un $K$-módulo. Supongamos que el polinomio
		mínimo $m(X)$ de $T$ es irreducible en $K[X]$. Demostrar que existen $K[X]$-submódulos simples $V_1,
		\dots, V_t$ de $V$ tal que $V = V_1 \bigoplus \dots \bigoplus V_t$ como $K$-módulo.
		\begin{proof}
			Dado que el polinomio mínimo $m(X)$ es irreducible $\Rightarrow$ el ideal $(m(X))$ es maximal, por
			tanto, se tiene que el cociente $K[X]/_{(m(X))}$ es un cuerpo que también denotaremos $F$.
			
			Como tenemos que $K[X] \rightarrow End_K(V)$, al aplicarle el Primer Teorema de Isomorfía, podemos
			descomponerlo como $K[X] \rightarrow K[X]/_{(m(X))} \hookrightarrow End_K(V)$, de lo que se concluye
			que $V$ es $F$-espacio vectorial.
			
			Ahora, dado que $V$ tiene una base finita como $K$-espacio vectorial y que $K \subseteq F$, tenemos
			que $V$ tiene un sistema de generadores finito como $F$-espacio vectorial del que, en virtud de un
			corolario visto en clase, podemos extraer un subconjunto $\{V_1, V_2, \dots, V_t\}$que genere una
			base, esto es, $V = V_1 \bigoplus V_2 \bigoplus \dots \bigoplus V_t$.
			
			Finalmente, por la forma de construcción de esta base, tenemos que sus elementos son isomorfos a
			$F$ y, por tanto, $K[X]$-submódulos simples.
		\end{proof}
		
		% Pág 19
		\item ** En las condiciones del ejercicio anterior, demostrar que el polinomio característico de $T$ es
		$m(X)^t$.
		\begin{proof}
			Empecemos porque $T$ debe satisfacer su ecuación característica por el Teorema de Cayley-Hamilton,
			y como $m(X)$ es un divisor de todo polinomio $p(X)$ que cumpla $T$, en particular, $m(X)$ es un
			divisor del polinomio característico de $T$.
			
			Veamos ahora que no sólo $m(X)$ divide al polinomio característico sino que es su único factor
			irreducible, esto es, el polinomio característico de $T$ es de la forma $(m(X))^n$ para algún $n
			\in \mathbb{N}$.
			
			Sea el polinomio $p(X)$ un factor irreducible del polinomio característico, entonces se tiene que
			tiene una raíz $\alpha$ en la clausura algebraica de $K$. Esto equivale a tener un vector propio
			$\lambda$ con coeficientes en la clausura de $K$ cumpliendo $T\lambda = \alpha\lambda$.
			
			Aplicando el polinomio mínimo al vector, tenemos que $0 = m(T)\lambda = m(\alpha)\lambda \Rightarrow
			\alpha$ es una raíz en la clausura algebraica de $m(X)$. Por tanto, $p(X)$ es un factor irreducible
			de $m(X)$, que también es irreducible, lo que nos lleva a que $m(X) = p(X) \Rightarrow$ el polinomio
			característico de $T$ sólo tiene a $m(X)$ como único factor irreducible, es decir, es de la forma
			$(m(X))^n$.
			
			Por último, como el grado del polinomio característico tiene el mismo valor que la dimensión del
			espacio en que se genera, calcularemos la $dim_K(V)$. Por la construcción que hicimos de $V$ con
			$V_1, V_2, \dots, V_t$ y que $V_i \cong F = K[X]/_{(m(X))} \Rightarrow dim_K(V) = t \cdot dim_K (V_i)
			= t \cdot dim_K(F) = t \cdot gr(m(X)) \Rightarrow (m(X))^t$ es el polinomio característico.
		\end{proof}
		
		% Pág 22
		\item ** Sea $R$ un álgebra sobre un cuerpo de característica distinta de 2, y $a, b, e \in R$
		idempotentes. Demostrar que si $e = a + b$, entonces $ab = ba = 0$. Si la característica es 2,
		encontrar un contraejemplo con $a \neq b$.
		\begin{proof}
			Si $e = a+b$ tenemos que $e^2 = (a+b)^2 = a^2 + b^2 + ab + ba = a + b + ab + ba = a+b \Rightarrow
			ab + ba = 0 \Rightarrow ba = -ab \Leftrightarrow ab = -ba$ y volviendo a utilizar que $a$ y $b$ son
			idempotentes, se llega a que $ab = a^2b = a(-ba) = -aba = (-ab)a = ba^2 = ba \Rightarrow 2ab = 2ba
			= ab + ba = 0$ pero como la característica del cuerpo no es 2, la única posibilidad es que $ab = ba
			= 0$.
			
			Como contraejemplo, nos iremos al álgebra $M_2(\mathbb{F}_2)$ y tomaremos $a = I_2$, $\displaystyle
			b = \begin{pmatrix} 1 & 0 \\ 0 & 0 \end{pmatrix}$, entonces $\displaystyle e = a+b = \begin{pmatrix}
			0 & 0 \\ 0 & 1 \end{pmatrix}$, claramente todas son idempotentes pero $ab = ba = b \neq 0_{2,2}$.
		\end{proof}
	
		% Pág 31
		\item * Calcular explícitamente una representación real no trivial de grado 2 del grupo de permutaciones
		$S_3$.
		\begin{proof}[Solución]
			Representaremos el grupo de las permutaciones $S_3$ como las aplicaciones sobre $\mathbb{R}^2$ que
			dejan invariantes los vértices de un triángulo equilátero como el de la imagen:
			
			\begin{figure}[H]
				\centering
				\includegraphics[width=5cm]{./3rd-roots-of-unity.png}
				\caption{Raíces cúbicas de la unidad. Imagen retocada de
				\href{https://upload.wikimedia.org/wikipedia/commons/3/3a/3rd-roots-of-unity.png}{Wikipedia}}
			\end{figure}
			
			En particular, tomaremos $A = W_3^0 = (0, 1)$, $\displaystyle B = W_3^2 = \left(\frac{-1}{2},
			\frac{\sqrt{3}}{2}\right)$ y $\displaystyle C = W_3^1 = \left(\frac{-1}{2}, \frac{-\sqrt{3}}{2}\right)$.
			Además, tomaremos $b=\{A, B\}$ como base de $\mathbb{R}^2$ notando que $C = -(A+B)$.
			
			Visto esto, nos basta ver que $A =_b (1, 0)$, $B =_b (0, 1)$ y $C =_b (-1, -1)$ y luego se calculan
			las operaciones de $\phi: S_3 \rightarrow GL\left(\mathbb{R}^2\right)$ como:
			$$\left.\begin{aligned}
				(1\ 2)A = A\\
				(1\ 2)B = B
			\end{aligned}\right\} \Rightarrow\phi(id) = \begin{pmatrix} 1 & 0\\ 0 & 1 \end{pmatrix}$$
			
			$$\left.\begin{aligned}
				(1\ 2)A = B\\
				(1\ 2)B = A
			\end{aligned}\right\} \Rightarrow \phi((1\ 2)) = \begin{pmatrix} 0 & 1\\ 1 & 0 \end{pmatrix}$$
			
			$$\left.\begin{aligned}
				(1\ 3)A = C\\
				(1\ 3)B = B
			\end{aligned}\right\} \Rightarrow \phi((1\ 3)) = \begin{pmatrix} -1 & 0\\ -1 & 1 \end{pmatrix}$$
			
			$$\left.\begin{aligned}
				(2\ 3)A = A\\
				(2\ 3)B = C
			\end{aligned}\right\} \Rightarrow \phi((2\ 3)) = \begin{pmatrix} 1 & -1\\ 0 & -1 \end{pmatrix}$$
			
			$$\left.\begin{aligned}
				(1\ 2\ 3)A = B\\
				(1\ 2\ 3)B = C
			\end{aligned}\right\} \Rightarrow \phi((1\ 2\ 3)) = \begin{pmatrix} 0 & -1\\ 1 & -1 \end{pmatrix}$$
			
			$$\left.\begin{aligned}
				(1\ 3\ 2)A = C\\
				(1\ 3\ 2)B = A
			\end{aligned}\right\} \Rightarrow \phi((1\ 3\ 2)) = \begin{pmatrix} -1 & 1\\ -1 & 0 \end{pmatrix}$$
		\end{proof}
       
		% Pág 43
		\item Si $r \in \mathbb{Q}$ es un entero algebraico, entonces $r \in \mathbb{Z}$\footnote{En el
		enunciado original, era una $q$ pero por conveniencia de notación, la he cambiado.}.
		\begin{proof}
			Supongamos que $r$ es un entero algebraico pero $r \notin \mathbb{Z}$ y sea $f(x) = x^n +
			a_{n-1}x^{n-1} + \dots + a_1x + a_0$ con $a_{n-1}, \dots, a_1, a_0 \in \mathbb{Z}$ el polinomio de
			coeficientes enteros del que $r$ es raíz, esto es, $f(r) = 0$.
			
			Como $r \in \mathbb{Q}$ podemos expresarlo como $\displaystyle r = \frac{p}{q}$ con $(p, q) = 1$,
			entonces nos queda que $\displaystyle f(r) = f\left(\frac{p}{q}\right) = \left(\frac{p}{q}\right)^n
			+ a_{n-1} \left(\frac{p}{q}\right)^{n-1} + \dots + a_1\frac{p}{q} + a_0 = 0$.
			
			Después, multiplicamos toda la expresión por $q^n$ para quitarnos los denominadores: $p^n +
			a_{n-1}qp^{n-1} + \dots + a_1q^{n-1}p + a_0q^n = 0$
			
			Llegados a este punto, vemos que $-p^n = q \cdot \left(a_{n-1}p^{n-1} + \dots + a_1q^{n-2}p +
			a_0q^{n-1}\right) \Rightarrow q$ divide a $p^n$ pero recordemos que $(p, q) = 1$.
		\end{proof}
	\end{enumerate}
\end{document}
